%   COMMANDS ZUM ZUSAMMENBAUEN DER KARTEN
%   ---------------------------------------

%   TikZ/PGF Settings f�r die Karten
\pgfmathsetmacro{\cardwidth}{6}
\pgfmathsetmacro{\cardheight}{8.5}
\pgfmathsetmacro{\imagewidth}{\cardwidth}
\pgfmathsetmacro{\imageheight}{0.75*\cardheight}
\pgfmathsetmacro{\stripwidth}{0.7}
\pgfmathsetmacro{\strippadding}{0.2}
\pgfmathsetmacro{\textpadding}{0.1}
\pgfmathsetmacro{\titley}{.94*\cardheight}
\pgfmathsetmacro{\titlex}{2.3}


\newcommand{\push}{\includegraphics[height = .6cm]{Push} }
\newcommand{\roam}{\includegraphics[height = .6cm]{Roam} }
\newcommand{\vision}{\includegraphics[height = .6cm]{Vision} }
\newcommand{\bigPush}{\includegraphics[height = 1cm]{Push} }
\newcommand{\bigRoam}{\includegraphics[height = 1cm]{Roam} }
\newcommand{\bigVision}{\includegraphics[height = 1cm]{Vision} }
\newcommand{\specialist}{\includegraphics[height = .5cm]{Ranged} }
\newcommand{\support}{\includegraphics[height = .5cm]{Siege} }
\newcommand{\assassin}{\includegraphics[height = .5cm]{Melee} }
\newcommand{\warrior}{\includegraphics[height = .5cm]{Flying} }
\newcommand{\vp}{\includegraphics[height = .6cm]{vp} }

\newcommand{\thisBattle}{\includegraphics[height = .6cm]{thisBattle} }
\newcommand{\adjBattle}{\includegraphics[height = .6cm]{adjBattle} }
\newcommand{\region}{\includegraphics[height = .6cm]{region} }
\newcommand{\unit}{\includegraphics[height = .6cm]{unit} }
\newcommand{\enters}{\includegraphics[height = .6cm]{enters} }
\newcommand{\strength}{\includegraphics[height = .6cm]{strength}}
\newcommand{\sorcery}{\includegraphics[height = .6cm]{sorcery}} 
\newcommand{\command}{\includegraphics[height = .6cm]{command}} 

%   Formen der einzelnen Kartenelemente/-bestandteile
\def\shapeCard{(0,0) rectangle (\cardwidth,\cardheight)}
\def\shapeLeftStripTop{(\strippadding,\cardheight/2+1) rectangle (\strippadding+\stripwidth,\cardheight-\strippadding-\strippadding-1)}
\def\shapeLeftStripBot{(\strippadding,-0.2) rectangle (\strippadding+\stripwidth,\cardheight/2-3)}
\def\shapeLeftStripShort{(\strippadding,\cardheight-\strippadding-1) rectangle (\strippadding+\stripwidth,\cardheight+0.2)}
\def\shapeRightStripShort{(\cardwidth-\stripwidth-\strippadding,\cardheight-\strippadding-1) rectangle (\cardwidth-\strippadding,\cardheight+0.2)}
\def\shapeTitleArea{(2*\strippadding+\stripwidth,\cardheight-\strippadding) rectangle (\cardwidth-2*\strippadding-\stripwidth,\cardheight-2*\stripwidth)}
\def\shapeContentArea{(2*\strippadding+\stripwidth,0.5*\cardheight) rectangle (\cardwidth+0.2,-0.2)}


%   Stylings f�r die Elemente definieren
\tikzset{
    %   runde Ecken f�r die Karten
    cardcorners/.style={
        rounded corners=0.2cm
    },
    %   runde Ecken f�r die "F�hnchen"
    elementcorners/.style={
        rounded corners=0.1cm
    }
}

\newcommand{\cardborder}{
    \draw[cardcorners] \shapeCard;
}

\newcommand{\cardcost}[1]{
		\node at (\cardwidth*.1,\cardheight*.8) {\textbf{\LARGE #1}};
}

\newcommand{\cardtier}[1]{
		\node at (\cardwidth*.1,\cardheight*.7) {\textbf{\LARGE #1}};
}

\newcommand{\cardbackground}[1]{
		\clip[cardcorners] \shapeCard;
		%\draw[fill=white] (\cardwidth-.2,\cardheight-.2)--(.2,\cardheight-.2)--(.2,.2)--(\cardwidth-.2,.2)--cycle;
    %\node at (\cardwidth/2, \cardheight/2) {\includegraphics[height = 9.5cm]{#1}};
		\node[rotate=270] at (5,6.15) {\textbf{\large{\uppercase{#1}}}};
		\node[rotate=0] at (5,8) {\includegraphics[height = .95cm]{#1}};
}

\newcommand{\cardtitle}[1]{
    \node[text width=4.8cm, below right] at (0.5,8.3) {
       \shadowtext{\parbox{2.8cm}{\begin{spacing}{1.2}\textbf{\hspace{-.5cm}\uppercase{\Large #1}}\end{spacing}}}
    };
}

\newcommand{\cardhp}[1]{
		\node[left] at (\cardwidth-\stripwidth*.45,\cardheight-5.4*\stripwidth) {\textbf{\LARGE #1} \Health};
}

\newcommand{\cardcontent}[1]{
   \node[draw, rectangle, ultra thick, text width=5.5cm, below] at (\cardwidth*.5, 0.36*\cardheight) {\textrm{\Large#1}};
		
}

\newcommand{\cardimage}[1]{
	\node at (\cardwidth*.4, \cardheight*.7) {\includegraphics[height = 2.5cm, width = 3cm, keepaspectratio]{#1}};
}

\newcommand{\cardvp}[1]{
		\node at (\cardwidth-.7,\cardheight-.5) {\textbf{\LARGE #1\vp}};
}

\newcommand{\first}[1]{
	\node at (.2*\cardwidth, .45*\cardheight) {\includegraphics[height=1cm]{#1}};
}
\newcommand{\second}[1]{
	\node at (.4*\cardwidth, .45*\cardheight) {\includegraphics[height=1cm]{#1}};
}
\newcommand{\third}[1]{
	\node at (.6*\cardwidth, .45*\cardheight) {\includegraphics[height=1cm]{#1}};
}
\newcommand{\fourth}[1]{
	\node at (.8*\cardwidth, .45*\cardheight) {\includegraphics[height=1cm]{#1}};
}

\newcommand{\general}[1]{
\ifthenelse{\equal{#1}{1}\OR \equal{#1}{2}}{\draw[thick] (-1,.8) -- (2.5, .8) -- (2.5, -1); \node at (1.25, .27) {\includegraphics[height=1cm]{Player}};}{}
\ifthenelse{\equal{#1}{2}}{\draw[thick] (3.5,-1) -- (3.5, .8) -- (7, .8);\node at (4.75, .27) {\includegraphics[height=1cm]{Player}};}{}
}
